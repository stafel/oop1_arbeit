\subsection[Funktion 2 JavaFX GUI]{Funktion 2 JavaFX als grafische Oberfläche eingesetzt mit mindestens 2 Views}

\subsubsection{Anforderung}

Der Szenarienbeschrieb beinhaltet drei Paare mit je einem Übersichts- und einem Detail-Screen im Total sechs Views.

\subsubsection{Design}

Für die Übersichts-Screens wurden ListViews geplant um eine einfache Ansicht in einem minimalistischen Design bereitzustellen mit Änderungs und Löschfunktionen direkt auf der Zeile.
Die Detail-Screens wurden als Erstellungs- und Bearbeitungs-Screen geplant welche je nach Aufruf Nodes ein- oder ausblenden aufgebaut auf VBox/HBox und Textboxen. Es wurde noch nicht festgelegt ob die Detail-Screens in einem neuen Fenster/Stage erscheinen oder die aktuelle Stage übernehmen sollten.

\subsubsection{Entwicklung}
Beim der Entwicklung des Referenz-Screen-Paars wurde klar, dass:
\begin{itemize}
	\item die ListView zu unflexibel ist für die verschiedenen Business-Objekte: TableViews stattdessen verwendet
	\item VBox/HBox-Verschachtelungen die Detail-Screens nur bedingt darstellen können: GridPanes wurden stattdessen verwendet
	\item die Stage übernahme des Detail-Screens vom Übersichtscreens die komplexität erhöht ohne grosse Vorteile: Detail-Views starten somit in eigener Stage
	\item die Views immer etwa gleich aufgebaut sind: eine BaseController-Klasse wurde eingeführt
	\item SubViews die Komplexität massiv erhöhen: keine SubScreens verwendet
\end{itemize}
Obwohl weitere Entkoppelungen der Nodes besonders im Overview-Screen möglich sind wurden diese wegen Zeitdruck nicht umgesetzt.

\subsubsection{Test}
\begin{itemize}
	\item Starten und Beenden der Views in eigenen Stages.
	\item Starten und Beenden der Views in Parent Stage.
	\item Parameterübergabe zwischen Controller.
	\item Aufruf von Events.
	\item Verändern von View-Feldern.
	\item Vererbung von Event und Feldverknüpfungen
\end{itemize}

\subsubsection{Fazit}
Die Arbeit mit FXML war besonders ermüdend, da kaum Debugginginformationen weitergereicht wurden.